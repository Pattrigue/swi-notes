\section{Essence Terminology}

\begin{longtable}{| >{\raggedright\arraybackslash}p{0.2\textwidth} | p{0.75\textwidth} |}
\hline
\textbf{Term} & \textbf{Definition} \\
\termdef{Digital innovation}{Products, services processes, or business models that emerge within or between business units and where digital technology is a key factor in triggering or enabling value.}

\termdef{Paradigm innovation}{Products, services, processes, or business models that emerge from radical changes in the mental models of what a business is; who the users are; or what the market is.}

\termdef{Position innovation}{Products, services, processes, or business models that emerge by fitting earlier solutions into new uses.}

\termdef{Process innovation}{New or radically improved ways to produce products or services.}

\termdef{Product innovation}{New or radically changed products or services.}

\termdef{Software-intensive system}{Systems where software plays a key role in providing functionality and value. A software-intensive system is a system where software is essential for the design, construction, or implementation of the system as a whole.}

\termdef{Appreciative system}{The set of values, preferences, and norms, in terms of which they make sense of practice situations, formulate goals and directions for action, and determine what constitutes professional conduct.}

\termdef{Apprehension}{Understandings - leaps of faith - without factual grounds for asserting its existence. When we apprehend, we understand an idea or notion without factual basis for its existence.}

\termdef{Assertion}{Beliefs or facts backed by evidence.}

\termdef{Backing}{Why the \termlink{Prospect} is an attractive and useful solution to the Problem.}

\termdef{Capability}{An ability or facility devised to handle \termlink{Manifestation}s as part of a Solution that Provides Value. \textit{Basically, what we call a feature normally.}}

\termdef{Contribution}{Column in the Configuration Table. What we design as part of a solution. Used to understand and devise how the constructive parts of a solution could be designed.}

\termdef{Criteria}{What to look for to determine if desired qualities are achieved.}

\termdef{Inner Environment}{The substance and organization of what we design and implement.}

\termdef{Outer Environment}{The surroundings in which our problem resides and where the contributions we design will operate. External services, implements, repositories, and people are examples of elements in the Outer Environment.}

\termdef{Inquiry}{Inquiry is a reflective thoughts process of understanding and resolving a problem. This starts the moment we recognize an \termlink{Indeterminate Situation} as problematic.}

\termdef{Keystones}{What we used to build the Solution - technologies, \termlink{Components}, information, and human resources are examples of keystones.}

\termdef{Leverage}{\termlink{Keystone}s of the \termlink{Inner Environment} that help create a great \termlink{Contribution}.}

\termdef{Manifestation}{An object, event, or action that reflects or gives a tangible or visible form to the Problem. Such objects, events, or actions, in turn, should be matched by \termlink{Capability} provided by the \termlink{Contribution} of the project.}

\termdef{Problem}{A Problem reflects an understanding of a situation. In Essence, Problems are used as analytical instruments. They are formulated as part of inquiring into a \termlink{Problematic Situation}. Problems, therefore, have no objective status.}

\termdef{Prospect}{Our overall \termlink{End-in-view}. What we believe would be a solution if the products developed in the project are put to proper use. A Prospect thereby reflects an understanding of the problem and an adequate way to solve it.}

\termdef{Qualification}{Do we solve the entire \termlink{Problem} or are there limitations in our resolution? Will our resolution be acceptable despite these limitations?}

\termdef{Rationale}{Row in the Configuration Table. The logical basis that makes actions meaningful and helps reason about strategy and tactics.}

\termdef{Reservation}{Limitations in the solution (why the \termlink{Inner Environment} might not support a complete solution to the \termlink{Problem}).}

\termdef{Rebuttal}{Reasons why limitations in the solution may be acceptable.}

\termdef{Resolution}{Why the \termlink{Problem} might be resolved by what is to be built, why this is important, and why the result is attractive. \termlink{Resolution} combines \termlink{Prospect}, \termlink{Warrant} and \termlink{Backing}.}

\termdef{Situation}{Column in the Configuration Table. The problem and the context of it. An understanding that reflects an underlying mental model of the problem and the system ecology, who the users are, and what they care for.}

\termdef{Solution}{Column in the Configuration Table. How our \termlink{Contribution} resolves the \termlink{Situation}. Solution is about the purpose and overall idea of the project.}

\termdef{Strategy}{Row in the Configuration Table. The master plan for solving the overall problem: Including the scope of the problem, the key \termlink{Components} for building the solution, and the qualifications regarding limitations and constraints that may affect the utility or acceptability of a solution.}

\termdef{Tactics}{Row in the Configuration Table. Actions planned in accordance with a strategy to achieve specific ends.}

\termdef{Valuation}{Column in the Configuration Table. Used to determine if we are moving in the right direction to resolve the \termlink{Problem}. Used to offer a range of ways to develop ideas, invent alternative lines of action, and, not least, provide criteria to assess and evaluate qualities of contributions to provide a sound basis for decision making.}

\termdef{Value}{Something desired as a (partial) solution to the \termlink{Problem}.}

\termdef{Warrant}{Reasons why solving the \termlink{Problem} is important.}

\termdef{Constituent}{The distinctions and relations we experience when we see a problem.}

\termdef{Element}{Core objects or events in the problem domain. Can be services, artifacts, repositories, individuals, groups, or events.}

\termdef{Event}{Events are objects in time; especially things of importance for the unfolding of the problem and where one or more objects are involved.}

\termdef{Existential}{Items can be existential (tangible) meaning that they have being in time and space or are experienced to be so.}

\termdef{Ideational}{Items may be ideational - not yet in existence - and represent ideas for actions, things to build, or hypothetical concepts for example regarding user characteristics.}

\termdef{Indeterminate Situation}{The condition before inquiry starts is called indeterminate. The elements of the situation are not determined - let alone understood - and the scope of inquiry is not presumed yet.}

\termdef{Determinate Situation}{The determinate situation is the final outcome of inquiry where uncertainty and doubtfulness is resolved and replaced by a closed, finished, and unified situation.}

\termdef{Object}{Transitory objects that can be represented - like any object - by nouns or substantives.  Events have substance and are characterized by a delimiting beginning, an interval, and a termination.}

\termdef{Problematic Situation}{A situation becomes problematic when our uncertainty and doubtfulness makes us stop and consider how to settle it. This is when we begin to consider which objects, events, and qualities seem relevant to the situation.}

\termdef{Idea}{Ideas describe something that may happen. The development and evaluation of ideas are a central part of the progressive determination of a problem and ways to address it. For that reason, ideas may be vague when we start working on a not-too-familiar context and develop as we gain insights.}

\termdef{Means}{Instruments used to attain an \termlink{End} - something that is useful in achieving a result.  A database, a server, an app, etc., could be ends in and by themselves, but as we design and build them, they might also be means of building a system to serve a higher end.}

\termdef{Material Means}{\termlink{Means} - including observed data, facts and artifacts - that combined serve to create a resolved situation.}

\termdef{Procedural Means}{Procedural means serve to determine how and if a resolved situation is attained.}

\termdef{Transaction}{Interactions between the inquirer and the materials at hand - the elements deemed part of the problematic situation and the means available for resolving it.}

\termdef{End}{The ultimate end for inquiry is to resolve a problematic situation via existential changes. Such an end represents a fulfilling close and termination of the project.}

\termdef{End-in-view}{An idea of an end to be reached and represents the purpose of taking a step towards an overall resolving end. The end-in-view is the anticipated consequence of taking a step - the existential change to be effected by that step.}

\termdef{Approriate}{A \termlink{Solution} is appropriate to the \termlink{Problem} if there is a meaningful match between the two.}

\termdef{Design By Contract}{An idea from software design. If stated preconditions are true, and if the software executes correctly, then the result will honor the stated postconditions. In Essence this idea is used to ensure that any activity in a project will leave the project in a stable state.}

\termdef{ETVX Model}{A model of development activities consisting of four parts: Entry criteria, Task, Validation, and eXit Criteria.}

\termdef{Pivot}{When one \termlink{Prospect} is replaced by another due to a significant change in project \termlink{Rationale}.}

\termdef{RST Review}{There are three focus areas for reviews in Essence: \termlink{Rationale}, \termlink{Strategy}, and \termlink{Tactics}, and accordingly we use the term RST Reviews to denote Essence Reviews.  The purpose of a RST Review is to determine if a Configuration Table needs to be revised.}

\termdef{Components}{Equipment, sensors, or actuators used actively as part of building a solution and accessed via crafted interfaces.}

\termdef{External Implements}{Equipment, sensors, or actuators available in the \termlink{Outer Environment} and accessed via trivial interfaces.}

\termdef{External People}{Specific persons, categories of people, stakeholders, authorities, social or professional networks, organizations, or others that will interface with the system.}

\termdef{External Repository}{Repository for collecting, storing, disseminating, and/or aggregating information via trivial interfaces.}

\termdef{External Services}{Services that are used without modification and reduce what is left for us to deal with. For example, providers of marketing information, tracking of equipment, maintenance, streaming services, digital infrastructures, and more.}

\termdef{Initial Problem}{The first intuitive idea of the problem and the situation in which it resides.}

\termdef{Internal Human Resources}{Specific users, categories of users, authorities, social or professional networks, organizations, or others that will have particular roles or functions designed as part of building a solution. Cyberhuman systems are prominent examples of systems where internal people are \termlink{Leverage} points}

\termdef{Internal Information}{Used for collecting, storing, disseminating and/or aggregating information to be accessed via trivial interfaces}

\termdef{Technology}{Scientific knowledge, machinery, and equipment on which a design might be based}

\termdef{Role}{Roles in Essence are for learning across problem and solution domains. Learning usually requires insights and experiences to be shared, and one principal way of sharing is through interaction among people. There are four Essence Roles: \termlink{Anchor Role}, \termlink{Challenger Role}, \termlink{Child Role}, and \termlink{Responder Role}.}

\termdef{Anchor Role}{The team member in charge of \termlink{Valuation}. A team captain (although not a team manager) who facilitates credible evaluations and provides a reasonable basis for decisions on whether to \termlink{Pivot} or persevere.}

\termdef{Challenger Role}{The team member who provides the reasoning and politics underlying the \termlink{Solution}. Aiming for a \termlink{Solution} with a good balance between \termlink{Problem} and \termlink{Contribution} in the specific \termlink{Situation}. Open to new ideas and good arguments.}

\termdef{Child Role}{Any team member exploring the \termlink{Situation}. Unorthodox and creative. Offers ideas and proposals without responsibility. Questions conventional views on uses, needs, users, and the application of technology. This Role is fleeting.}

\termdef{Responder Role}{Any developer working on the \termlink{Contribution}. Offers technological responses to challenges - always with a view to technological options and affordances that come up during the design process.}

\termdef{Prospect Scenarios}{Alternative ideas for defining a problem and ways to solve it. They are used to develop prospective definitions of a problem and corresponding approaches to solving it.}

\termdef{Contribution Scenario}{The part of a \termlink{Prospect Scenario} aiming to explore conceivable \termlink{Contributions}.}

\termdef{Situation Scenario}{The part of a \termlink{Prospect Scenario} aiming to explore a \termlink{Problematic Situation}.}

\termdef{Icon}{Symbolic representations such as images to illustrate key qualities in a \termlink{Solution}.}

\termdef{Metaphor}{Metaphors can represent design principles via analogies using figurative language, symbols, slogans, and similar. The idea usually is to provoke ideas by looking at principles employed in solutions to problems that somehow are similar to the problem at hand. Such principles could then perhaps inspire solutions for the current project.}

\termdef{Proposition}{A Proposition aims to express core properties of the resolution of the problem.}

\termdef{Prototype}{A Prototype suggests a physical design and indicates how to operate it.}

\hline
\end{longtable}